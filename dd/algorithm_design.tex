\section{Taxi queue management}
%TODO Alex

\section{Taxi sharing matching}
%TODO Alex

\section{Taxi fee splitting}
This algorithm is run by the back-end in the case of shared rides to compute the percentages of the taxi fee that each passenger has to pay.

The algorithm computes the fee percentages proportionally to the overall distance traveled by all the people (travelers) that each passenger on the taxi brings with himself.

The input data is:
\begin{itemize}
    \item $P$: list of passengers;
    \item $T_i$: list of travellers associated with passenger $i$;
    \item $d(t)$: distance traveled by traveler $t$.
\end{itemize}

The algorithm computes $f_i$, that is the percentage of the fee that the passenger $i$ has to pay.
$f_i$ is expressed as a decimal number, i.e. $0 \le f_i \le 1$.

The algorithm works as follows:

\begin{equation}
    f_i = \dfrac
        {\displaystyle \sum_{t \in T_i} d(t)}
        {\displaystyle \sum_{p \in P} \sum_{t \in T_p} d(t)}
\end{equation}

An example of an application of this algorithm is shown in~\autoref{fee-splitting-ex}.

\begin{table}
\begin{center}
\begin{tabular}{ l  l  l  l  l }
    \hline
    \textbf{Passenger} & \textbf{Traveler} & \multicolumn{2}{c}{\textbf{Distance}} & \textbf{Fee} \\
    \hline
    \multirow{3}{*}{Alice} & 1 & 90 & \multirow{3}{*}{115} & \multirow{3}{*}{64\%} \\
    & 2 & 20 & & \\
    & 3 & 5 & & \\
    \hline
    \multirow{3}{*}{Bob} & 1 & 25 & \multirow{3}{*}{65} & \multirow{3}{*}{36\%} \\
    & 2 & 30 & & \\
    & 3 & 10 & & \\
    \hline
    \textbf{Total} & & \textbf{180} & \textbf{180} & \textbf{100\%} \\
    \hline
\end{tabular}
\caption{An example of the taxi fee splitting algorithm.}
\label{fee-splitting-ex}
\end{center}
\end{table}

\section{Taxi waiting calculation}
%TODO Eleonora
