\section{Taxi queue management}
The algorithm is run every time there is a new event which changes the order or the number of the elements in one of the system queues.

According to the type of event, the algorithm updates the queues moving, deleting or creating elements. There are several types of events:
\begin{enumerate}
	\item New request incomes \label{event:new_req}
	\item Taxi Driver accepts a request \label{event:accepted_req}
	\item Taxi Driver refuses a request \label{event:refuses_req}
	\item Taxi Driver changes zone \label{event:changed_zone}
	\item Taxi Driver changes his/her status \label{event:changed_status}
\end{enumerate}
The input data are:
\begin{itemize}
	\item $Q_i$: list of FIFO queues, where $i$ is the zone of the correspondent queue 
	\item $e$: event which will change the queues
	\item $OOS$: list of taxi drivers out of service
	\item $B$: list of busy taxi drivers
	\item $P$: list of taxi drivers with a pending ride
\end{itemize}  

The algorithm works differently accordingly with the input event. The alternatives are explained in the following subsections.

\subsection{Type \ref{event:new_req}, new request incomes}
The algorithm extracts the zone $z$ of the new incoming request, pops the first element of the $Q_z$ and inserts this element in $P$ list.

\subsection{Type \ref{event:accepted_req}, taxi driver accepts a request}
The algorithm extracts the taxi driver $t$ from the event and removes him/her from $P$. After that it inserts $t$ into $B$.

\subsection{Type \ref{event:refuses_req}, taxi driver refuses a request}
The algorithm extracts the taxi driver $t$ from the event and it removes him/her from $P$. After that it retrieves the taxi driver's actual zone $z$ and it pushes $t$ in $Q_z$.

\subsection{Type \ref{event:changed_zone}, taxi driver changes zone}
The algorithm extracts the taxi driver $t$, the previous zone $pz$ and the next zone $nz$ from the event and removes $t$ from $Q_{pz}$ and pushes $t$ into $Q_{nz}$.

\subsection{Type \ref{event:changed_status}, taxi driver changes status}
If the taxi driver $t$ changes his/her status from “not in service” to “in service”, the algorithm extracts the zone $z$ from the event, it removes $t$ from $OOS$ and pushes $t$ into $Q_z$.

If the taxi driver $t$ changes his/her status from “in service” to “not in service”, the algorithm extracts the zone $z$ from the event, it removes $t$ from $Q_z$ and pushes $t$ into $OOS$.


\section{Taxi sharing matching}
%TODO Alex

\section{Taxi fee splitting}
This algorithm is run by the back-end in the case of shared rides to compute the percentages of the taxi fee that each passenger has to pay.

The algorithm computes the fee percentages proportionally to the overall distance traveled by all the people (travelers) that each passenger on the taxi brings with himself.

The input data are:
\begin{itemize}
    \item $P$: list of passengers;
    \item $T_i$: list of travellers associated with passenger $i$;
    \item $d(t)$: distance traveled by traveler $t$.
\end{itemize}

The algorithm computes $f_i$, that is the percentage of the fee that the passenger $i$ has to pay.
$f_i$ is expressed as a decimal number, i.e. $0 \le f_i \le 1$.

The algorithm works as follows:

\begin{equation}
    f_i = \dfrac
        {\displaystyle \sum_{t \in T_i} d(t)}
        {\displaystyle \sum_{p \in P} \sum_{t \in T_p} d(t)}
\end{equation}

An example of an application of this algorithm is shown in~\autoref{fee-splitting-ex}.

\begin{table}
\begin{center}
\begin{tabular}{ l  l  l  l  l }
    \hline
    \textbf{Passenger} & \textbf{Traveler} & \multicolumn{2}{c}{\textbf{Distance}} & \textbf{Fee} \\
    \hline
    \multirow{3}{*}{Alice} & 1 & 90 & \multirow{3}{*}{115} & \multirow{3}{*}{64\%} \\
    & 2 & 20 & & \\
    & 3 & 5 & & \\
    \hline
    \multirow{3}{*}{Bob} & 1 & 25 & \multirow{3}{*}{65} & \multirow{3}{*}{36\%} \\
    & 2 & 30 & & \\
    & 3 & 10 & & \\
    \hline
    \textbf{Total} & & \textbf{180} & \textbf{180} & \textbf{100\%} \\
    \hline
\end{tabular}
\caption{An example of the taxi fee splitting algorithm.}
\label{fee-splitting-ex}
\end{center}
\end{table}

\section{Taxi waiting calculation}
%TODO Eleonora
