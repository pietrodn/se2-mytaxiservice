\section{Component interfaces}
\label{sec:component-interfaces}

\subsection{Application server to database}
The application server communicates with the DBMS via the Java Persistence API over standard network protocols. Thus, the DB and the application server layers can be deployed on different tiers, as well on the same one.

The low-level technicalities about the specific dialect of SQL for the selected DBMS are abstracted by the Java Persistence API, which also deals with the O/R mapping.

\subsection{Application server to front-ends (REST API)}
\label{sec:rest-api}
The front-ends of the system (the web application and the mobile app) shall communicate with the application server using the back-end programmatic interface described in the RASD~\cite{rasd} and implemented as a RESTful interface over the HTTPS protocol.
The RESTful interface is implemented in the application server using JAX-RS and uses XML as the data representation language.

The detailed REST API implemented by the core system and by the plugins is described in the following pages.

The conventions used for defining the REST API are the following:
\begin{enumerate}
    \item Each Session Bean can offer some functionalities as part of the public API.
    \item Plugins can add other functionalities and extend the existing ones.
    \item Functions can return values and one or more errors.
    \item \returns{Return values} are highlighted in red.
    \item \plugin{Extensions} by plugins are highlighted in blue.
    \item Position data consists in an ordered tuple containing two floating point values (latitude and longitude).
    \item Features can be available to all users (\textbf{A}), logged in users (\textbf{U}), passengers (\textbf{P}) or taxi drivers (\textbf{D}) only.
    \item The features only available to logged in users require a login token, provided by the \texttt{login} function.
\end{enumerate}

\subsubsection{UserManager}
Functions implemented by \textbf{UserManager} (\autoref{tab:rest-usermanager}):
\begin{description}
    \item[\texttt{login}] This function allows any registered user to log into the system using his username and password. If the credentials are correct, the function returns a token to be used in the future requests to identify the user. Otherwise, an error is returned.

    \item[\texttt{register}] This function creates a new user in the system with the provided data. This function can create both passengers and taxi drivers. If the entered data is correct, an email is sent to the user address to confirm the email address.

    \item[\texttt{confirm\_email}] This function confirms the email address of a newly registered user using the token sent by email after the registration.

    \item[\texttt{delete\_user}] This function deletes the user account with all the related data. As an extra security measure, username and password are required even if the user is logged in.

    \item[\texttt{edit\_profile}] This function allows users to edit their profile information. If a new email is entered, the same confirmation process of \texttt{register} will be followed.
\end{description}

\begin{table}
    \centering
    \begin{small}
    \begin{tabular}{l l l p{0.4\textwidth}}
        \textbf{Service} & \textbf{Users} & \multicolumn{2}{l}{\textbf{Parameters and return values}} \\
        \hline
        %% COMMON THINGS
        \multirow{2}{*}{All services} & \multirow{2}{*}{A} & \texttt{token} & Authentication token. \\
        & & \texttt{\returns{errors}} & List of errors.\\
        \hline
        % LOGIN
        \multirow{3}{*}{\texttt{login}} & \multirow{2}{*}{A} & \texttt{user\_name} & Username of the registered user. \\
        && \texttt{password} & Password of the user. \\
        && \texttt{\returns{token}} & Authentication token to be used in future requests.\\
        \hline
        % REGISTER
        \multirow{4}{*}{\texttt{register}} & \multirow{4}{*}{A} & \texttt{user\_name} & Username to register. \\
        && \texttt{password} & Password chosen by the user. \\
        && \texttt{email} & E-mail address chosen by the user. \\
        && \texttt{type} & \texttt{passenger | driver} \\
        \hline
        % CONFIRM_EMAIL
        \multirow{2}{*}{\texttt{confirm\_email}} & \multirow{2}{*}{U} & \texttt{user\_name} & Username of the registered user. \\
        && \texttt{email\_token} & Confirmation token sent by e-mail. \\
        \hline
        % DELETE_USER
        \multirow{2}{*}{\texttt{delete\_user}} & \multirow{2}{*}{U} & \texttt{user\_name} & Username of the registered user \\
        && \texttt{password} & Password of the user. \\
        \hline
        % EDIT_PROFILE
        \multirow{4}{*}{\texttt{edit\_profile}} & \multirow{4}{*}{U} & \texttt{user\_name} & New username. \\
        && \texttt{password} & New password. \\
        && \texttt{email} & New email. \\
        && \multicolumn{2}{c}{one parameter for each profile information.}\\
        \hline
    \end{tabular}
    \end{small}
    \caption{REST API implemented by the UserManager bean.}
    \label{tab:rest-usermanager}
\end{table}

\subsubsection{RideManager}
Functions implemented by \textbf{RideManager} (\autoref{tab:rest-ridemanager}):
\begin{description}
    \item[\texttt{get\_ride\_info}] This function returns all the information about a ride. A user can obtain information about a ride if and only if one of these conditions is met:
    \begin{enumerate}
        \item the user is one of the passengers of the ride;
        \item the user is the taxi driver whom the ride is assigned;
        \item the ride is shared and new passengers can join.
    \end{enumerate}
    An error is returned if the ride ID is invalid or if the user is not authorized to access the data.

    The function fetches the following information for a ride: the number of passengers, the number of travelers, the origin and destination of the ride (if available) and the ride status. The status of the ride can be: reserved (the ride has been reserved but there is not yet any taxi driver allocated to it), waiting (the taxi driver is going to meet the passenger), running (passengers on board), done (the ride ended).

    \item[\texttt{request\_taxi}] This function allows a passenger to call a taxi. The passenger has to specify its position along with the number of travelers, and if he wants to share the ride. If the sharing feature is enabled, the passenger has to provide a destination. If the taxi call is successful, the function returns the ID of the newly created ride. An error is returned if the data is not valid or if the passenger is currently in a ride or waiting for a taxi.

    The taxi sharing feature is added by the ride sharing plugin.

    \item[\texttt{update\_ride\_status}] This function is used by the taxi driver client to send messages about a ride. The taxi manager can signal that he has picked up all the passenger or that the ride is ended. If the message if inconsistent with the current status of the ride, an error is returned.

    This function can be extended with other types of messages or events that can be sent by the taxi driver.
\end{description}

\begin{table}
    \centering
    \begin{small}
    \begin{tabular}{l l l p{0.4\textwidth}}
        \textbf{Service} & \textbf{Users} & \multicolumn{2}{l}{\textbf{Parameters and return values}} \\
        \hline
        % COMMON THINGS
        \multirow{2}{*}{All services} & \multirow{2}{*}{A} & \texttt{token} & Authentication token. \\
        & & \texttt{\returns{errors}} & List of errors.\\
        \hline
        % UPDATE_RIDE_STATUS
        \multirow{3}{*}{\texttt{update\_ride\_status}} & \multirow{3}{*}{D} & \texttt{ride\_id} & ID of the ride.\\
        & & \texttt{ride\_event} & \texttt{passengers\_on\_board | ride\_finished} \\
        & & \texttt{\returns{ride\_info}} & The output of \texttt{get\_ride\_info} for the ride \texttt{ride\_id}.\\
        \hline
        % GET_RIDE_INFO
        \multirow{7}{*}{\texttt{get\_ride\_info}} & \multirow{7}{*}{U} & \texttt{ride\_id} & ID of the ride returned by \texttt{request\_taxi}. \\
        & & \texttt{\returns{origin}} & Position of the starting point.\\
        & & \texttt{\returns{destinations}} & Positions of the destinations.\\
        & & \texttt{\returns{num\_travelers}} & Total number of travellers.\\
        & & \texttt{\plugin{num\_passengers}} & Total number of passengers.\\
        & & \texttt{\returns{status}} & \texttt{reserved | waiting | running | done}\\
        & & \texttt{\returns{wait\_time}} & Estimated waiting time in seconds. Returns -1 if there is no taxi to wait for.\\
        \hline
        % REQUEST TAXI
        \multirow{5}{*}{\texttt{request\_taxi}} & \multirow{5}{*}{P} & \texttt{origin} & Position of the passenger.\\
        & & \texttt{travelers} & The number of travelers.\\
        & & \texttt{\plugin{destination}} & Destination of the ride. Only required if \texttt{sharing\_enabled=true}.\\
        & & \texttt{\plugin{sharing\_enabled}} & \texttt{true | false}\\
        & & \texttt{\returns{ride\_id}} & An ID number identifying the ride.\\
        \hline
    \end{tabular}
    \end{small}
    \caption{REST API implemented by the RideManager bean.}
    \label{tab:rest-ridemanager}
\end{table}

\subsubsection{TaxiManager}
Functions implemented by \textbf{TaxiManager} (\autoref{tab:rest-taximanager}):
\begin{description}
    \item[\texttt{update\_status}] This function is used by the mobile client of taxi drivers to update the status of the driver. The taxi driver can become unavailable only if he is not currently assigned to a ride.
    \item[\texttt{update\_position}] This function is used by the mobile client of taxi drivers to periodically update the position of the driver.
\end{description}

\begin{table}
    \centering
    \begin{small}
    \begin{tabular}{l l l p{0.5\textwidth}}
        \textbf{Service} &  \textbf{Users} & \multicolumn{2}{l}{\textbf{Parameters and return values}} \\
        \hline
        % COMMON THINGS
        \multirow{2}{*}{All services} & \multirow{2}{*}{A} & \texttt{token} & Authentication token. \\
        & & \texttt{\returns{errors}} & List of errors.\\
        \hline
        % UPDATE_STATUS
        \multirow{1}{*}{\texttt{update\_status}} & \multirow{1}{*}{D} & \texttt{status} & \texttt{unavailable | available | busy}\\
        \hline
        % UPDATE_POSITION
        \multirow{1}{*}{\texttt{update\_position}} & \multirow{1}{*}{D} & \texttt{position} & The position of the taxi driver.\\
        \hline
    \end{tabular}
    \end{small}
    \caption{REST API implemented by the TaxiManager bean.}
    \label{tab:rest-taximanager}
\end{table}

\subsubsection{HistoryManager}
Functions implemented by \textbf{HistoryManager} (\autoref{tab:rest-HistoryManager}):
\begin{description}
    \item[\texttt{get\_past\_rides}] This function returns the information (in the same format of \texttt{get\_ride\_info}) about the rides in which the user was involved (as a taxi driver or a passenger).
\end{description}

\begin{table}
    \centering
    \begin{small}
    \begin{tabular}{l l l p{0.5\textwidth}}
        \textbf{Service} &  \textbf{Users} & \multicolumn{2}{l}{\textbf{Parameters and return values}} \\
        \hline
        % COMMON THINGS
        \multirow{2}{*}{All services} & \multirow{2}{*}{A} & \texttt{token} & Authentication token. \\
        & & \texttt{\returns{errors}} & List of errors.\\
        \hline
        % GET_PAST_RIDES
        \multirow{1}{*}{\texttt{get\_past\_rides}} & \multirow{1}{*}{U} & \texttt{\returns{rides}} & List of rides in the \texttt{get\_ride\_info} format.\\
        \hline
    \end{tabular}
    \end{small}
    \caption{REST API implemented by the HistoryManager bean.}
    \label{tab:rest-HistoryManager}
\end{table}

\subsubsection{Plugins}
Functions implemented by the \textbf{plugins} \emph{ride sharing} and \emph{ride reservation} (\autoref{tab:rest-plugins}):
\begin{description}
    \item[\texttt{reserve\_taxi}] This function is implemented by the \emph{ride reservation} plugin and allows any passenger to reserve a ride for a future time. The submitted timestamp must be between 2h and 48h after the current time. If the sharing feature is enabled, the passenger has to provide a destination.
    \item[\texttt{list\_shared\_rides}] This function is implemented by the \emph{ride sharing} plugin and lists all the shared rides compatible with the chosen destination, date and time, and number of travellers. If the data is valid, the function outputs a list with the feasible ride information in the same format of the \texttt{get\_ride\_info} request. If no feasible ride is found, the function returns the empty list. If the data is invalid, an error is returned.
    \item[\texttt{join\_shared\_ride}] This functions is implemented by the \emph{ride sharing} plugin and allows a passenger to join a shared ride, adding him to the list of passengers of the ride and adding his destination to the ride destinations. A passenger can join a ride only if the ride is ``feasible'' as defined by the algorithm described in~\autoref{sec:taxi-sharing-matching}. The passenger has to provide a destination, along with the number of passengers and the ID of the ride he wants to join. Feasible rides can be obtained by calling \texttt{list\_shared\_rides}. If the provided data is not valid or the ride is not feasible, an error is returned. If the request is successful, the function returns the ride information in the same format of \texttt{get\_ride\_info}.

    \item[\texttt{get\_ride\_fee}] This function is implemented by the \emph{ride sharing} plugin and allows the taxi driver and the passengers of the ride to know the percentage of the taxi fee that each passenger has to pay, according to the algorithm described in~\autoref{sec:taxi-fee-splitting}.

    The function returns a dictionary containing as keys the passengers' usernames and as values the percentage of the fee that each one has to pay, expressed as a floating point number from 0 to 1. If the function is called on a ride that has only one passenger, the fee percentage for that passenger will be 1.

    This information is only available to the taxi driver and to the passengers of the ride.
\end{description}

\begin{table}
    \centering
    \begin{small}
    \begin{tabular}{l l l p{0.4\textwidth}}
        \textbf{Service} & \textbf{Users} & \multicolumn{2}{l}{\textbf{Parameters and return values}} \\
        \hline
        % COMMON THINGS
        \multirow{2}{*}{All services} & \multirow{2}{*}{A} & \texttt{token} & Authentication token. \\
        & & \texttt{\returns{errors}} & List of errors.\\
        \hline
        % RESERVE_TAXI
        \multirow{6}{*}{\texttt{reserve\_taxi}} & \multirow{6}{*}{P} & \texttt{origin} & The position of the starting point of the ride.\\
        && \texttt{destination} & The position of the passenger's destination.\\
        && \texttt{travelers} & The number of travellers.\\
        && \texttt{time} & Time of arrival of the taxi (ISO 8601 format).\\
        && \texttt{\plugin{sharing\_enabled}} & \texttt{true | false}\\
        && \texttt{\returns{ride\_id}} & An ID number identifying the ride.\\
        \hline
        % LIST_SHARED_RIDES
        \multirow{5}{*}{\texttt{list\_shared\_rides}} & \multirow{5}{*}{P} & \texttt{origin} & The position of the starting point.\\
        && \texttt{destination} & The position of the passenger's destination.\\
        && \texttt{travelers} & The number of travellers.\\
        && \texttt{time} & Time of arrival of the taxi (ISO 8601 format).\\
        && \texttt{\returns{rides}} & A dictionary containing a list of feasible rides, with the same format of \texttt{get\_ride\_info}.\\
        \hline
        % JOIN_SHARED_RIDE
        \multirow{4}{*}{\texttt{join\_shared\_ride}} & \multirow{4}{*}{P} & \texttt{ride\_id} & The ID of the ride.\\
        && \texttt{destination} & The position of the passenger's destination.\\
        && \texttt{\returns{ride\_info}} & The output of \texttt{get\_ride\_info} for the ride \texttt{ride\_id}.\\
        \hline
        % GET_RIDE_FEE
        \multirow{2}{*}{\texttt{get\_ride\_fee}} & \multirow{2}{*}{U} & \texttt{ride\_id} & The ID of the ride.\\
        && \texttt{\returns{fees}} & Fee percentage for each user, as a dictionary \texttt{\{username $\Rightarrow$ Float\}}.\\
        \hline
    \end{tabular}
    \end{small}
    \caption{REST API implemented by the plugins \emph{taxi reservation} and \emph{taxi sharing}.}
    \label{tab:rest-plugins}
\end{table}

\subsection{Configuration file (application server)}
\label{sec:config-file}
The application server is configurable by means of a XML configuration file.
The configuration file contains the following settings:
\begin{itemize}
    \item the boundaries of the taxi zones, expressed as polygons (list of coordinates);
    \item the credentials of the user that can access the database;
    \item the host, port and name of the database;
    \item the network settings of the application server (listening port, host, ...);
    \item any other settings that will be useful in the implementation phase.
\end{itemize}

\subsection{Web server to browser}
\label{sec:server-to-browser}
The users' browsers communicate with the web server via HTTPS requests. Any unencrypted request will be denied, as stated in the RASD~\cite{rasd}.

\subsection{Plug-in interface}
\label{sec:plug-in-interface}
The application server exposes extension points to be used by plug-ins: they are the only points in which the plug-ins can access the system. For example, they may not query the database layer directly.
Plug-ins may add new services to the REST API of the application server.

Plug-in declaration, discovery and activation are made explicit by means of manifest files and plug-in registry.
