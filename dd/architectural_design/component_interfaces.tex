\section{Component interfaces}
\label{sec:component-interfaces}

\subsection{Application server to database}
The application server communicates to the DB via JPA and JDBC over standard network protocols. Thus, the DB and the application server layers can be deployed on different tiers, as well on the same one.

The low-level technicalities about the specific dialect of SQL for the selected DBMS are abstracted by the Java Persistence API, which also deals with the O/R mapping.

\subsection{Application server to front-ends}
The front-ends of the system (the web application and the mobile app) shall communicate with the application server using the same back-end programmatic inteface described in the RASD and implemented as a RESTful interface over the HTTPS protocol.

The RESTful interface is implemented in the application server using JAX-RS.

Also, if the web server fails for any reason, the back-end interface will be accessible by the users by means of the mobile application.

The detailed REST API implemented by the core application server is described in~\autoref{tab:rest-core}. The extensions implemented by the plugins to be developed are defined in~\autoref{tab:rest-plugins}.

\begin{table}
    \centering
    \begin{small}
    \begin{tabular}{l l p{0.5\textwidth}}
        \textbf{Service} & \multicolumn{2}{l}{\textbf{Parameters and return values}} \\
        \hline
        % COMMON THINGS
        \multirow{2}{*}{Common to all services} & \texttt{token} & The authentication token returned by \texttt{login}, used for mantaining the session of the users. \newline Not needed for \texttt{login} and \texttt{registration}. \\
        & \texttt{\returns{errors}} & A dictionary (id $\Rightarrow$ message) of errors returned by the operation. If the operation is successful, it's empty.\\
        \hline
        % LOGIN
        \multirow{3}{*}{\texttt{login}} & \texttt{user\_name} & The username of the registered user \\
        & \texttt{password} & The password of the user. \\
        & \texttt{\returns{token}} & The authentication token to be used in future requests.\\
        \hline
        % REGISTER
        \multirow{4}{*}{\texttt{register}} & \texttt{user\_name} & The username to register. \\
        & \texttt{password} & The password chosen by the user. \\
        & \texttt{email} & The e-mail address chosen by the user. \\
        & \texttt{type} & \texttt{passenger | driver} \\
        \hline
        % UPDATE_STATUS
        \multirow{3}{*}{\texttt{update\_status}} & \texttt{longitude} & The longitude coordinate of the taxi driver.\\
        & \texttt{latitude} & The latitude coordinate of the taxi driver.\\
        & \texttt{status} & \texttt{unavailable | available}\\
        \hline
        % GET_RIDE_INFO
        \multirow{8}{*}{\texttt{get\_ride\_info}} & \texttt{ride} & The ID of the ride returned by \texttt{request\_taxi}. \\
        & \texttt{\returns{ride\_id}} & The ride ID.\\
        & \texttt{\returns{origin}} & Dictionary containing the coordinates of the starting point.\\
        & \texttt{\returns{destinations}} & List of dictionaries containing the coordinates of the destinations.\\
        & \texttt{\returns{num\_travelers}} & The total number of travelers.\\
        & \texttt{\plugin{num\_passengers}} & The total number of passengers (return value).\\
        & \texttt{\returns{status}} & \texttt{reserved | waiting | running | done}\\
        & \texttt{\returns{wait\_time}} & The estimate waiting time in seconds. Returns -1 if there's no taxi to wait for.\\
        \hline
        % REQUEST TAXI
        \multirow{4}{*}{\texttt{request\_taxi}} & \texttt{longitude} & The longitude coordinate of the passenger.\\
        & \texttt{latitude} & The latitude coordinate of the passenger.\\
        & \texttt{travelers} & The number of travelers.\\
        & \texttt{\plugin{sharing\_enabled}} & \texttt{true | false}\\
        & \texttt{\returns{ride\_id}} & An ID number identifying the ride.\\
        \hline
    \end{tabular}
    \end{small}
    \caption{REST API implemented by the core application layer. \returns{Return values} are highlighted. \plugin{Extensions} by plugins are colored blue.}
    \label{tab:rest-core}
\end{table}

\begin{table}
    \centering
    \begin{small}
    \begin{tabular}{l l p{0.5\textwidth}}
        \textbf{Service} & \multicolumn{2}{l}{\textbf{Parameters and return values}} \\
        \hline
        % COMMON THINGS
        \multirow{2}{*}{Common to all services} & \texttt{token} & The authentication token returned by \texttt{login}, used for mantaining the session of the users. \newline Not needed for \texttt{login} and \texttt{registration}. \\
        & \texttt{\returns{errors}} & A dictionary (id $\Rightarrow$ message) of errors returned by the operation. If the operation is successful, it's empty.\\
        \hline
        % RESERVE_TAXI
        \multirow{8}{*}{\texttt{reserve\_taxi}} & \texttt{origin\_longitude} & The longitude coordinate of the passenger.\\
        & \texttt{origin\_latitude} & The latitude coordinate of the passenger.\\
        & \texttt{destination\_longitude} & The longitude coordinate of the passenger's destination.\\
        & \texttt{destination\_latitude} & The latitude coordinate of the passenger's destination.\\
        & \texttt{travelers} & The number of travelers.\\
        & \texttt{time} & Time of arrival of the taxi (ISO 8601 format).\\
        & \texttt{\plugin{sharing\_enabled}} & \texttt{true | false}\\
        & \texttt{\returns{ride\_id}} & An ID number identifying the ride.\\
        \hline
        % LIST_SHARED_RIDES
        \multirow{5}{*}{\texttt{list\_shared\_rides}} & \texttt{origin\_longitude} & The longitude coordinate of the passenger.\\
        & \texttt{origin\_latitude} & The latitude coordinate of the passenger.\\
        & \texttt{destination\_longitude} & The longitude coordinate of the passenger's destination.\\
        & \texttt{destination\_latitude} & The latitude coordinate of the passenger's destination.\\
        & \texttt{travelers} & The number of travelers.\\
        & \texttt{time} & Time of arrival of the taxi (ISO 8601 format).\\
        & \texttt{\returns{rides}} & A dictionary containing a list of feasible rides, with the same format of \texttt{get\_ride\_info}.\\
        \hline
        % JOIN_SHARED_RIDE
        \multirow{5}{*}{\texttt{join\_shared\_ride}} & \texttt{ride\_id} & The ID of the ride.\\
        & \texttt{longitude} & The longitude coordinate of the destination.\\
        & \texttt{latitude} & The latitude coordinate of the destination.\\
        & \texttt{\returns{ride\_info}} & If the request is successful, the output of \texttt{get\_ride\_info} for the ride \texttt{ride\_id}.\\
        \hline
    \end{tabular}
    \end{small}
    \caption{REST API implemented by the plugins \emph{taxi reservation} and \emph{taxi sharing}. \returns{Return values} are highlighted. \plugin{Extensions} by plugins are colored blue.}
    \label{tab:rest-plugins}
\end{table}

\subsection{Web server to browser}
The users' browsers communicate with the web server via HTTPS requests. Any unencrypted request will be denied, as stated in the RASD.

\subsection{Plug-in interface}
The application server exposes extension points to be used by plug-ins: they're the only points in which the plug-ins can access the system. For example, they may not query the database layer directly.
Plug-ins may add new services to the REST API of the application server.

Plug-in declaration, discovery and activation are made explicit by means of manifest files and plug-in registry.
