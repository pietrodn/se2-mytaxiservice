\section{Taxi waiting calculation}
When a user requests a taxi and his request is assigned to a taxi driver, the user receives the estimated time of arrival of the taxi. This delay is computed by using the Google Maps API. \cite{Google-Maps-Directions-API}
\begin{enumerate}
\item The user requests a taxi sending his current position to the application server
\item The request is accepted by a taxi driver, who also sends his current position to the application server
\item \label{point-a} The application server computes the waiting time sending a request to google maps, the format of the request is specified in \autoref{tab:gmaps-parameters} 
\item \label{point-b}The application server sends to the passenger the estimated time.
\item Points \ref{point-a} , \ref{point-b} are repeated every 90 seconds until the taxi notifies that it has picked up the passenger
\end{enumerate}

\begin{table}
\begin{center}
\begin{tabular}{l p{0.5\textwidth}}
\hline
URL & \url{http://maps.googleapis.com/maps/api/directions/xml}\\
\hline
\texttt{origin} & GPS coordinates of the taxi driver\\
\hline
\texttt{destination} & GPS coordinates of the passenger\\
\hline
\texttt{key} & API key\\
\hline
\texttt{mode} & \texttt{driving}\\
\hline
\texttt{departure\_time} & \texttt{now}\\
\hline
\texttt{traffic\_model} & \texttt{pessimistic}\\
\hline
\end{tabular}
\caption{Parameters of the Google Maps API call for the estimation of the waiting time.}
\label{tab:gmaps-parameters}
\end{center}
\end{table}



