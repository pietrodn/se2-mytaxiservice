All the decisions in the DD have been taken following functional and nonfunctional requirements written in the RASD, although some changes are explained in \autoref{RASD-changes}.

\section{Components and functional requirements}

The following table \autoref{tab:Comp-Table} show the {\bf section in the RASD of the requirement} fulfilled by a specific {\bf component in DD}. 


\begin{table}[h]
\begin{center}
\begin{tabular}{|p{0.5\textwidth}|p{0.5\textwidth}|}
\hline
{\bf Component}  & {\bf Requirements}\\
\hline
EmailSender & 3.2.1 the e-mail confirmation process in User registration \\
\hline
UserManager & 3.2.1 User registration, 3.2.2 User login, 3.2.8 User profile management.\\
\hline
HistoryManager & 3.2.8 rides history in User profile management.\\
\hline
TaxiManager & 3.2.4 Ride request notification to the taxi driver, 3.2.5 Taxi availability handling.\\
\hline
TaxiQueueManager & 3.2.5 Taxi availability handling (taxi queue management) \\
\hline
TaxiQueue & 3.2.5 Taxi availability handling (taxi queue)\\
\hline
RideManager & 3.2.3 Standard taxi call.\\
\hline 
ReservedRidePlugin & 3.2.6 Taxi reservation.\\
\hline
SharedRidePlugin & 3.2.7 Ride sharing.\\
\hline
\end{tabular}
\caption{Table that links components to functional requirements}
\label{tab:Comp-Table}
\end{center}
\end{table}

\section{Other requirements}


The user non-functional requirements that are common to all clients interfaces are not explained in this document, because they will be observed during the implementation part; although some of them, written in the following list, are shown in the UI Design \autoref{ch:ui-design}.

\begin{itemize}
\item The first screen must ask the user to login.
\item The dashboard with links to every function shall be displayed in the home page.
\item The top bar must show the last taxi service called.
\item The possibility to choose the language at all times.
\end{itemize}

The other {\bf requirements in RASD} fulfilled by a {\bf DD section} are shown in table \autoref{non-func-req}


%TODO GPS on the mobile app???


\begin{table}[h]
\begin{center}
\begin{tabular}{|p{0.5\textwidth}|p{0.5\textwidth}|}
\hline
{\bf Section}  & {\bf Requirements}\\
\hline
\autoref{sec:component-database} Component view: Database & 
3.1.3 Software interfaces\\
\hline
\autoref{sec:comp-view-web-server} Web server & 3.4.4 Maintainability: MVC pattern \\
\hline
\autoref{sec:styles-patterns} Selected architectural styles and patterns &  3.4.4 Maintainability: MVC pattern, 3.1.3 Software interfaces: modularity \\
\hline
\autoref{sec:rest-api} Application server to front-ends (REST API) & 3.2.9 Programmatic interface \\
\hline
\autoref{sec:config-file} Configuration file (application server) & 3.1.1 User interfaces: Server back-end \\
\hline
\autoref{sec:server-to-browser} Web server to browser & 3.1.4 communications interfaces \\
\hline
\autoref{sec:plug-in-interface} Plug-in interface & 3.1.3 Software interfaces: modularity, 3.2.9 Programmatic interface \\
\hline

\end{tabular}
\caption{Table that links sections to non-functional requirements}
\label{non-func-req}
\end{center}
\end{table}

\section{RASD modifications}
\label{RASD-changes}
Java SE8 used back-end has been changed into \textcolor{red}{Java EE} as can be seen in sections: \ref{sec:component-view}, \ref{sec:deployment-view}.








