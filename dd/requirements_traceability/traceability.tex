All the decisions in the DD have been taken following functional and nonfunctional requirements written in the RASD.
In the making of the Design Document, some changes were made to the RASD. They are explained in \autoref{sec:RASD-changes}.

\section{Functional requirements and components}

\autoref{tab:Comp-Table} maps the functional requirements contained in \textbf{specific sections} of the RASD to the \textbf{components} in the Design Document that fulfill them.

\begin{table}[h]
\begin{center}
\begin{tabular}{|l|p{0.7\textwidth}|}
\hline
{\bf Component (DD)}  & {\bf Requirements (RASD)}\\
\hline
EmailSender & 3.2.1 e-mail confirmation process in User registration \\
\hline
UserManager & 3.2.1 User registration \newline 3.2.2 User login \newline 3.2.8 User profile management.\\
\hline
HistoryManager & 3.2.8 rides history in User profile management.\\
\hline
TaxiManager & 3.2.4 Ride request notification to the taxi driver \newline 3.2.5 Taxi availability handling.\\
\hline
TaxiQueueManager & 3.2.5 Taxi availability handling (taxi queue management) \\
\hline
TaxiQueue & 3.2.5 Taxi availability handling (taxi queue)\\
\hline
RideManager & 3.2.3 Standard taxi call.\\
\hline
ReservedRidePlugin & 3.2.6 Taxi reservation.\\
\hline
SharedRidePlugin & 3.2.7 Ride sharing.\\
\hline
\end{tabular}
\caption{Table that links components to functional requirements.}
\label{tab:Comp-Table}
\end{center}
\end{table}

\section{Non-functional requirements}
The non-functional UI requirements that are common to all client interfaces will be mainly followed during the implementation part.
Some of them, though, are shown in \autoref{ch:ui-design} (UI Design):

\begin{itemize}
\item The first screen must ask the user to login.
\item The dashboard with links to every function shall be displayed in the home page.
\item The top bar must show the last taxi service called.
\item The possibility to choose the language at all times.
\end{itemize}

The other functional (programmatic interface) and non-functional {\bf requirements in RASD} fulfilled by the Design Document are shown in \autoref{non-func-req}.


\begin{table}[h]
\begin{center}
\begin{tabular}{|p{0.5\textwidth}|p{0.5\textwidth}|}
\hline
{\bf Section (DD)}  & {\bf Requirements (RASD)}\\
\hline
Component view: Database (\ref{sec:component-database}) &
3.1.3 Software interfaces: MySQL \\
\hline
Web server (\ref{sec:comp-view-web-server}) & 3.1.3 Software interfaces \newline
3.4.4 Maintainability: MVC pattern \newline
3.4.5 Portability, see \autoref{sec:RASD-changes} \\
\hline
Mobile client (\ref{sec:mobile-client})	& 3.1.2 Hardware interfaces\\
\hline
Selected architectural styles and patterns (\ref{sec:styles-patterns}) &  3.4.4 Maintainability: MVC pattern \newline
3.1.3 Software interfaces: modularity \\
\hline
Application server to front-ends (REST API) (\ref{sec:rest-api}) &  \textcolor{blue}{3.2.9 Programmatic interface} \newline
3.1.4 Communications interfaces \newline
3.4.3 Security: HTTPS \\
\hline
Configuration file (application server) (\ref{sec:config-file}) & 3.1.1 User interfaces: Server back-end \\
\hline
Web server to browser (\ref{sec:server-to-browser}) & 3.1.4 Communications interfaces \newline
3.4.3 Security: HTTPS \\
\hline
Plug-in interface (\ref{sec:plug-in-interface}) & \textcolor{blue}{3.2.9 Programmatic interface} \newline
3.1.3 Software interfaces: modularity \\
\hline
Other design decisions (\ref{sec:other-design-decisions}) & 3.4.2 Availability \newline
3.4.3  Security: user's password in DB hashed and salted \\
\hline
Application server: ER diagram (\ref{fig:er-diagram}) & 3.4.3 Security: user's password in DB hashed and salted \\
\hline

\end{tabular}
\caption{Table that links sections to functional and non-functional requirements, the functional requirement is written in  \textcolor{blue}{blue}.}
\label{non-func-req}
\end{center}
\end{table}

\section{Modifications to the RASD}
\label{sec:RASD-changes}

The RASD incorrectly stated that the back-end would be written using the Java SE 8 framework; it has been changed into \textcolor{red}{Java EE 7} as can be seen in sections \ref{sec:component-view}, \ref{sec:deployment-view}.

The following requirements have been {\bf removed from the RASD} as they are deemed unnecessary.
\begin{itemize}
\item Web application point 2 in 3.1.1 User interfaces: the web pages must be accessible also by text-only browsers.
\item Point 4 in 3.4.3 Security: The system must log all login attempts with IP addresses for at least 7 days.
\item Point 5 in 3.4.3 Security: The modules must state clearly which data they will need to read and write.
\end{itemize}
