\section{Project risks}

The project risks are listed below. An evaluation of those risks is provided in~\autoref{tab:project-risks}.

\begin{description}
    \item[Delays over the expected deadlines:] the project could require more time than expected. If this happens, we could release a first, incomplete but working version (e.g. without the web interface or the plugins for taxi reservation and taxi sharing) and build the less essential features later.

    \item[Lack of communication among team members:] the team often works remotely and this can lead to misunderstandings in fundamental decisions, to conflicts over the division of the work among team members and to conflicts in code versioning. Those difficulties can be overcome by explicitly defining (e.g. in this document) the responsibilities of each team member, and by writing clear and complete specification and design documents.

    \item[Lack of experience in programming with the specific frameworks:] the team has no actual experience in programming using Java EE. This will certainly slow down the development.

    \item[Requirements change:] the requirements may change during the development in unexpected way. This risk can't be prevented, but can be mitigated by writing reusable and extensible software.
\end{description}

\begin{table}[p]
\centering
    \begin{tabular}{| l | l | l |}
        \hline
        \textbf{Risk}           & \textbf{Probability}  & \textbf{Effects}  \\
        \hline
        Delays                  & High                  & Moderate          \\
        \hline
        Lack of communication   & Low                   & Moderate          \\
        \hline
        Lack of experience      & Certain               & Moderate          \\
        \hline
        Requirements change     & Very low              & Moderate          \\
        \hline
    \end{tabular}
    \caption{Evaluation of project risks.}
    \label{tab:project-risks}
\end{table}
