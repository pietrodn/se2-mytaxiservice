\chapter{Function Points estimation}
\label{chap:function-points}

\section{Internal Logic Files}
An Internal Logic File (ILF) is defined as a \emph{homogeneous set of data used
and managed by the application}.

\section{External Interface Files}
An External Interface File (EIF) is defined as a \emph{homogeneous set of data used by the application but generated and maintained by other applications}.

\section{External Input}
An External Input (EI) is defined as an \emph{elementary operation to elaborate data coming form the external environment}.

\section{External Output}
An External Output (EO) is defined ad an \emph{elementary operation that generates data for the external environment}. It usually includes the elaboration of data from logic files.

\section{External Inquiry}
An External Inquiry (EQ) is defined as an \emph{elementary operation that involves input and output, without significant elaboration of data from logic files}.

\section{Complexity levels of Function Points}
\autoref{tab:weight-estimation} shows the estimation of weights of Function Points, based on the number of data elements. The tables are taken from~\cite{cocomo-manual}.

\begin{table}[h]
\centering
\begin{subtable}{\textwidth}
    \centering
    \begin{tabular}{| c | l | l | l |}
        \hline
         & \multicolumn{3}{c|}{\textbf{Data Elements}} \\
        \hline
        \textbf{Record Elements} & 1-19 & 20-50 & 51+ \\
        \hline
        1       & Low     & Low     & Avg.     \\
        2-5     & Low     & Avg.    & High     \\
        6+      & Avg.    & High    & High     \\
        \hline
    \end{tabular}
    \caption{Weight estimation for ILFs and EIFs.}
\end{subtable}

\vspace{2em}

\begin{subtable}{\textwidth}
    \centering
    \begin{tabular}{| c | l | l | l |}
        \hline
         & \multicolumn{3}{c|}{\textbf{Data Elements}} \\
        \hline
        \textbf{Record Elements} & 1-5 & 6-19 & 20+ \\
        \hline
        0-1     & Low     & Low     & Avg.     \\
        2-3     & Low     & Avg.    & High     \\
        4+      & Avg.    & High    & High     \\
        \hline
    \end{tabular}
    \caption{Weight estimation for EOs and EQs}
\end{subtable}

\vspace{2em}

\begin{subtable}{\textwidth}
    \centering
    \begin{tabular}{| c | l | l | l |}
        \hline
         & \multicolumn{3}{c|}{\textbf{Data Elements}} \\
        \hline
        \textbf{Record Elements} & 1-4 & 5-15 & 16+ \\
        \hline
        1       & Low     & Low     & Avg.     \\
        2-3     & Low     & Avg.    & High     \\
        3+      & Avg.    & High    & High     \\
        \hline
    \end{tabular}
    \caption{Weight estimation for EIs}
\end{subtable}
\caption{Estimation of weights for different types of Function Points.}
\label{tab:weight-estimation}
\end{table}

\section{Weights of Function Points}
The scores of all types of function points, by their type and weight, is shown in~\autoref{tab:fp-weights}. The table is taken from~\cite{cocomo-manual}.
\begin{table}
    \centering
    \begin{tabular}{| l | l | l | l |}
        \hline
        \multirow{2}{*}{\textbf{Function Type}} & \multicolumn{3}{c|}{\textbf{Complexity-Weight}} \\
        \cline{2-4}
        & Simple & Medium & Complex \\
        \hline
        External Input          & 3     & 4     & 6     \\
        External Output         & 4     & 5     & 7     \\
        External Inquiry        & 3     & 4     & 6     \\
        Internal Logic File     & 7     & 10    & 15    \\
        External Interface File & 5     & 7     & 10    \\
        \hline
    \end{tabular}
    \caption{Scores of function points by type and weight.}
    \label{tab:fp-weights}
\end{table}

\section{UFP to SLOC conversion}
We need the multiplicator to convert the Unadjusted Function Points to Source Lines of Code.
From~\cite{cocomo-manual}, the multiplicator for Java code is \textbf{53}.

\section{Count of FPs, by type and weight}
% Here we need to count the function points of our system.
% This is a table that summarizes them.

The total count of Function Points of our system is shown in~\autoref{tab:fp-count}.
\begin{table}[h]
    \centering
    \begin{tabular}{| l | l | l | l | l |}
        \hline
        \multirow{2}{*}{\textbf{Function Type}} & \multicolumn{3}{c|}{\textbf{Complexity-Weight}} & \multirow{2}{*}{\textbf{Total Points}} \\
        \cline{2-4}
        & Simple & Medium & Complex & \\
        \hline
        External Input          & $0 \cdot 3$     & $0 \cdot 4$     & $0 \cdot 6$     & 0     \\
        External Output         & $0 \cdot 4$     & $0 \cdot 5$     & $0 \cdot 7$     & 0     \\
        External Inquiry        & $0 \cdot 3$     & $0 \cdot 4$     & $0 \cdot 6$     & 0     \\
        Internal Logic File     & $0 \cdot 7$     & $0 \cdot 10$    & $0 \cdot 15$    & 0     \\
        External Interface File & $0 \cdot 5$     & $0 \cdot 7$     & $0 \cdot 10$    & 0     \\
        \hline
        \textbf{Total Points} & 0 & 0 & 0 & \textbf{0} \\
        \hline
        \textbf{SLOC} & \multicolumn{3}{|c|}{$\times 53$} & \textbf{0}\\
        \hline
    \end{tabular}
    \caption{Count of Function Points of our system.}
    \label{tab:fp-count}
\end{table}
