\section{Complexity levels of Function Points}
\autoref{tab:weight-estimation} how to evaluate the weights of Function Points, based on the number of data elements (from~\cite{cocomo-manual}).

\begin{table}[h]
\centering
\begin{subtable}{\textwidth}
    \centering
    \begin{tabular}{| c | l | l | l |}
        \hline
         & \multicolumn{3}{c|}{\textbf{Data Elements}} \\
        \hline
        \textbf{Record Elements} & 1-19 & 20-50 & 51+ \\
        \hline
        1       & Low     & Low     & Avg.     \\
        2-5     & Low     & Avg.    & High     \\
        6+      & Avg.    & High    & High     \\
        \hline
    \end{tabular}
    \caption{Weight estimation for ILFs and EIFs.}
\end{subtable}

\vspace{2em}

\begin{subtable}{\textwidth}
    \centering
    \begin{tabular}{| c | l | l | l |}
        \hline
         & \multicolumn{3}{c|}{\textbf{Data Elements}} \\
        \hline
        \textbf{Record Elements} & 1-5 & 6-19 & 20+ \\
        \hline
        0-1     & Low     & Low     & Avg.     \\
        2-3     & Low     & Avg.    & High     \\
        4+      & Avg.    & High    & High     \\
        \hline
    \end{tabular}
    \caption{Weight estimation for EOs and EQs}
\end{subtable}

\vspace{2em}

\begin{subtable}{\textwidth}
    \centering
    \begin{tabular}{| c | l | l | l |}
        \hline
         & \multicolumn{3}{c|}{\textbf{Data Elements}} \\
        \hline
        \textbf{Record Elements} & 1-4 & 5-15 & 16+ \\
        \hline
        1       & Low     & Low     & Avg.     \\
        2-3     & Low     & Avg.    & High     \\
        3+      & Avg.    & High    & High     \\
        \hline
    \end{tabular}
    \caption{Weight estimation for EIs}
\end{subtable}
\caption{Estimation of weights for different types of Function Points.}
\label{tab:weight-estimation}
\end{table}

According to these criteria, \autoref{tab:computed-weights} shows the weights for the functions of the system.

\begin{table}[h]
\centering
\begin{subtable}{\textwidth}
    \centering
    \begin{tabular}{| l | l |}
        \hline
        \textbf{Functions} & \textbf{Weights} \\
        \hline
        User & High\\
        Passenger & Avg\\
        TaxiDriver & High\\
        TaxiLog & High\\
        Ride & High\\
        \hline
    \end{tabular}
    \caption{Computed weights for ILFs}
\end{subtable}

\vspace{2em}

\begin{subtable}{\textwidth}
    \centering
    \begin{tabular}{| l | l |}
        \hline
        \textbf{Functions} & \textbf{Weights} \\
        \hline
        Maps & High\\
        \hline
    \end{tabular}
    \caption{Computed weights for EIFs}
\end{subtable}

\vspace{2em}

\begin{subtable}{\textwidth}
    \centering
    \begin{tabular}{| l | l |}
        \hline
        \textbf{Functions} & \textbf{Weights} \\
        \hline
        User registration & Avg\\
        User profile management & Low\\
        User login & Low\\
        Standard taxi call & High\\
        Taxi availability & Low\\
        Taxi reservation & High\\
        \hline
    \end{tabular}
    \caption{Computed weights for EIs}
\end{subtable}

\vspace{2em}

\begin{subtable}{\textwidth}
    \centering
    \begin{tabular}{| l | l |}
        \hline
        \textbf{Functions} & \textbf{Weights} \\
        \hline
        Notification to users & Low\\
        Confirmation mails & Low\\
        Notifications to taxi drivers & Low\\
        \hline
    \end{tabular}
    \caption{Computed weights for EOs}
\end{subtable}

\vspace{2em}

\begin{subtable}{\textwidth}
    \centering
    \begin{tabular}{| l | l |}
        \hline
        \textbf{Functions} & \textbf{Weights} \\
        \hline
        Taxi driver ride request & Low\\
        Ride sharing & High\\
        \hline
    \end{tabular}
    \caption{Computed weights for EQs}
\end{subtable}
\caption{The evaluation of the weights of the functions according to \autoref{tab:weight-estimation}.}
\label{tab:computed-weights}
\end{table}
