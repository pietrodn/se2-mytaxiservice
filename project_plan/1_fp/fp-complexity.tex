\section{Complexity levels of Function Points}
\autoref{tab:weight-estimation} shows the estimation of weights of Function Points, based on the number of data elements. The tables are taken from~\cite{cocomo-manual}.

\begin{table}[h]
\centering
\begin{subtable}{\textwidth}
    \centering
    \begin{tabular}{| c | l | l | l |}
        \hline
         & \multicolumn{3}{c|}{\textbf{Data Elements}} \\
        \hline
        \textbf{Record Elements} & 1-19 & 20-50 & 51+ \\
        \hline
        1       & Low     & Low     & Avg.     \\
        2-5     & Low     & Avg.    & High     \\
        6+      & Avg.    & High    & High     \\
        \hline
    \end{tabular}
    \caption{Weight estimation for ILFs and EIFs.}
\end{subtable}

\vspace{2em}

\begin{subtable}{\textwidth}
    \centering
    \begin{tabular}{| c | l | l | l |}
        \hline
         & \multicolumn{3}{c|}{\textbf{Data Elements}} \\
        \hline
        \textbf{Record Elements} & 1-5 & 6-19 & 20+ \\
        \hline
        0-1     & Low     & Low     & Avg.     \\
        2-3     & Low     & Avg.    & High     \\
        4+      & Avg.    & High    & High     \\
        \hline
    \end{tabular}
    \caption{Weight estimation for EOs and EQs}
\end{subtable}

\vspace{2em}

\begin{subtable}{\textwidth}
    \centering
    \begin{tabular}{| c | l | l | l |}
        \hline
         & \multicolumn{3}{c|}{\textbf{Data Elements}} \\
        \hline
        \textbf{Record Elements} & 1-4 & 5-15 & 16+ \\
        \hline
        1       & Low     & Low     & Avg.     \\
        2-3     & Low     & Avg.    & High     \\
        3+      & Avg.    & High    & High     \\
        \hline
    \end{tabular}
    \caption{Weight estimation for EIs}
\end{subtable}
\caption{Estimation of weights for different types of Function Points.}
\label{tab:weight-estimation}
\end{table}
