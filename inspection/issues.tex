This chapter contains the results of the code inspection that we did on the assigned classes and methods. All the points of the checklist~\cite{se-assignment} were checked, and we also found other bad practices not listed in the checklist.

\section{Notation}
\begin{itemize}
    \item The items of the code inspection checklist~\cite{se-assignment} will be referred as follows: \checklist{1}, \checklist{2}, \ldots
    \item A specific line of code will be referred as follows: \codeline{1234}.
    \item An interval of lines of code will be referred as follows: \codelines{1234}{1289}.
\end{itemize}

\section{Issues}
\label{sec:issues}

\subsection{CoordinatorLog class}
\label{sec:coordinatorlog-class}
\begin{enumerate}
	\item \checklist{7} There is a constant in the class which does not follow naming convention at \codeline{110}.
    \item \checklist{12} Blank lines at the beginning and at the end of method bodies are not consistent throughout the class.
    Missing blank line after ending of method block after \checklist{1606}.
    \item \checklist{23} The CoordinatorLog constructor at \codeline{206} is missing the JavaDoc.
    All the duplicated methods with the \texttt{logPath} parameter (see \checklist{27}) are missing the JavaDoc.
    \item \checklist{25} There are several problems in the class declaration.
    \begin{itemize}
        \item  Commented out code at \codeline{112}.
        \item Static and non-static attributes are interleaved in the CoordinatorLog declaration (\codelines{107}{150}).
        \item The visibility order of attributes is not respected (for instance \codelines{143}{144}).
        \item A constructor is declared in the middle of normal methods (\codeline{324}).
        \item There is a static block used for initialization at \codeline{1104}.
        \item Static and non-static methods are mixed.
    \end{itemize}
    \item \checklist{28} Two attributes are of raw type \texttt{Hashtable}. Generics could be used to ensure type safety and to clarify the code.

    \item \checklist{27} The class is full of duplicated code, which is very bad for maintainability. There are couples of methods that differ only for the presence of the \texttt{logPath} parameter:
    \begin{enumerate}
        \item \texttt{reUse()} at \codeline{264}, \codeline{288}
    	\item \texttt{setLocalTID()} at \codeline{865}, \codeline{879}
    	\item \texttt{openLog()} at \codeline{1127}, \codeline{1188}
    	\item \texttt{getLogged()} at \codeline{1248}, \codeline{1317}
    	\item \texttt{addLog()} at \codeline{1388}, \codeline{1399}
    	\item \texttt{removeLog()} at \codeline{1418}, \codeline{1465}
    	\item \texttt{keypoint()} at \codeline{1525}, \codeline{1607}
    	\item \texttt{finalizeAll()} at \codeline{1720}, \codeline{1759}
    	\item \texttt{startKeypoint()} at \codeline{1807}, \codeline{1874}
    \end{enumerate}
    This is also highlighted in the comment at \codeline{153}:
    \begin{quote}
        All the methods which take "String logPath" as parameter are same as the ones with out that parameter. These methods are added for delegated recovery support
    \end{quote}
    The couples of methods respectively differ for a single line; where the \texttt{logPath} argument is not present, then we have:
    \begin{quote}
        \small\texttt{CoordinatorLogStateHolder logStateHolder = defaultLogStateHolder;}
    \end{quote}
    When the \texttt{logPath} argument is present, instead we have:
    \begin{quote}
        \small\texttt{CoordinatorLogStateHolder logStateHolder = getStateHolder(logPath);}
    \end{quote}
    A feasible solution to remove this ugly duplication would be to edit the methods without the additional argument, making them call the other method with \texttt{logPath} equal to null. There, a simple conditional would handle both the cases.

    \item In the CoordinatorLog class, the name \texttt{logPath} is used both for a private attribute for the class (\codeline{134}) and as an argument in many methods which are clearly not setters (see above). This is a bad practice and generates confusion.
    \item The \texttt{CoordinatorLog.java} file contains 4 classes (CoordinatorLog, CoordinatorLogStateHolder, CoordinatorLogSection, SectionPool): they should be split in separate files.
\end{enumerate}
\subsubsection{finalizeAll()}
Two similar functions are assigned:
\begin{enumerate}
    \item \texttt{\footnotesize synchronized static void finalizeAll()}
    \item \texttt{\footnotesize synchronized static void finalizeAll(String logPath)}
\end{enumerate}

As stated in~\autoref{sec:coordinatorlog-class}, these two methods are almost identical except for the first line of code, so they will be analyzed together.

\begin{enumerate}
    \item \checklist{8} Indentation with 4 spaces is used everywhere except for \codeline{1721}, \codeline{1760}.
    \item \checklist{11} No braces for single-statement blocks at \codeline{1736}, \codeline{1742}, \codeline{1747}, \codeline{1775}, \codeline{1781}, \codeline{1786}.
    \item \checklist{17} Indentation is wrong (\codeline{1721}, \codeline{1760}).
    \item \checklist{36} \texttt{LogFile.close()} boolean return value is discarded (\codeline{1742}, \codeline{1781}).
\end{enumerate}

\subsubsection{startKeypoint()}
Two similar functions are assigned:
\begin{enumerate}
    \item \texttt{\footnotesize synchronized static boolean startKeypoint( LogLSN keypointStartLSN )}
    \item \texttt{\footnotesize synchronized static boolean startKeypoint( LogLSN keypointStartLSN, String logPath )}
\end{enumerate}

As stated in~\autoref{sec:coordinatorlog-class}, these two methods are almost identical except for the first line of code, so they will be analyzed together.

\begin{enumerate}
    \item \checklist{11} No braces for single-statement blocks at \codeline{1830}, \codeline{1897}.
    \item \checklist{44} Brutish programming at \codelines{1858}{1865}, \codelines{1925}{1932}: a sequence of characters is assigned using an array of bytes. A better solution would use the native Java method \texttt{String.getBytes()}.
    \item \checklist{27} Duplicated code into the two \texttt{startKeyPoint} methods.
\end{enumerate}


\subsubsection{dump()}
The main problem of this method is that it does nothing.
Indeed:
\begin{enumerate}
    \item the method has no return value;
    \item the method only sets some local variables;
    \item all the called methods have no side-effects;
    \item the method is not required by any interfaces implemented by the CoordinatorLog class.
\end{enumerate}
Actions are replaced with comments which shortly explain what should be done.
This method should be either removed or modified.
\begin{enumerate}
	\item \checklist{13} Line length exceeds 80 characters at \codeline{1968} with 86 characters and at \codeline{1981} with 94 characters.
	\item \checklist{19} Code commented at \codeline{1968} without date and with trivial motivation.
	\item \checklist{23} The Java code is incomplete and the specification is generic, although the function does nothing.
\end{enumerate}

\subsection{CoordinatorLogSection class}
\label{sec:coordinatorlogsection-class}

\begin{enumerate}
    \item The clause \texttt{extends java.lang.Object} (\codeline{2078}) in the class declaration is superfluous and should be eliminated.
    \item \checklist{28} Four attributes are of raw type \texttt{Vector}. Java Generics could be used to ensure type safety and to clarify the code.
\end{enumerate}

\subsubsection{doFinalize()}
\begin{enumerate}
    \item \checklist{11} Braces are not used around the following one-statement blocks: \codeline{2108}, \codeline{2111}, \codeline{2114}, \codeline{2117}.
    \item \checklist{17} Inconsistent alignment before the equality sign at \codeline{2120}.
    \item \checklist{18} Missing comments in method body.
    \item \checklist{27} All the if statements are identical to the ones in the method \texttt{synchronized void reUse()} at \codeline{2145}.
\end{enumerate}

\subsubsection{reUse()}
\begin{enumerate}
    \item \checklist{8} Wrong indentation at \codeline{2163}. Inconsistent alignment at \codeline{2159}.
    \item \checklist{9} Tabs are used for indentation at \codeline{2163}.
    \item \checklist{11} Braces are not used around the following one-statement blocks: \codeline{2147}, \codeline{2150}, \codeline{2153}, \codeline{2156}.
    \item \checklist{17} Inconsistent alignment before the equality sign at \codeline{2159}.
    \item \checklist{18} Missing comments in method body.
\end{enumerate}

\subsection{RecoveryManager class}
\label{sec:recoverymanager-class}

\begin{enumerate}
	\item \checklist{19} There is a lot of commented code for example in the body of the \texttt{shutdown} method at \codeline{746}. There are also some methods entirely commented out:
	\begin{itemize}
		\item \codelines{1305}{1349}
		\item \codelines{1759}{1816}
	\end{itemize}
	\item \checklist{28} Four attributes (\codelines{145}{157}) are of raw type \texttt{Hashtable}. Generics could be used to ensure type safety and to clarify the code.
    \item There is a static attribute \texttt{lockObject} (\codeline{167}) used as a lock object for the entire class. This is not really necessary since a lock can be acquired on the Class object.\footnote{The Java™ Tutorials: Intrinsic Locks and Synchronization (\url{https://docs.oracle.com/javase/tutorial/essential/concurrency/locksync.html})}
    \item There is a \emph{TODO} within a commented code block at \codeline{930} which should be checked.
\end{enumerate}

\subsubsection{initialise()}
\begin{enumerate}
    \item \checklist{9} Tabs are used for indentation at \codelines{195}{199}, \codeline{210}.
    \item \checklist{13} If tabs are considered as one character every line respects it, but if a tab is considered as 4 characters (like 4 spaces)  it is not respected at \codeline{210}.
    \item \checklist{10} Braces are used with the ``Kernighan and Ritchie'' style, but at \codeline{196} ``Allman style'' is followed.
    \item \checklist{19} Commented out code is not motivated and doesn't contain a date (\codeline{200}).
    \item \checklist{23} JavaDoc for this method is too generic.
    The comments don't explain well the code: for exaple at \codeline{191} it is not clear what the comment is referred to and why there is the the commented out code at line \codeline{200}.
    \item \checklist{53} At \codeline{222} the catch of the exception executes the \texttt{exc.printStackTrace();} instruction: probably this is not the most appropriate behaviour.
\end{enumerate}
