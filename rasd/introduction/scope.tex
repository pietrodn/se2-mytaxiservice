The system is a taxi reservation and dispatching system for large cities. Its aim is to simplify the access of passengers to the service and to guarantee a fair management of taxi queues.

The system consists in a back-end server application (\emph{myTaxi Server}), a web application front-end (\emph{myTaxi Web}) and in a mobile application (\emph{myTaxi Mobile}).

The system has 2 types of users: passengers and taxi drivers; it should allow the users to sign up and login with their credentials.
The system has to know the location of both the passengers and the taxi drivers.

The system allows any passenger to request a taxi, informing him o her of the incoming taxi code and the estimated waiting time.

The system knows about the available taxi drivers and, when a request is incoming, informs one of them about the location of the available passenger; the taxi driver can either accept or deny the ride.
If the taxi driver accepts the ride, the system sends a confirmation to the passenger, together with the estimated waiting time.
If the taxi driver rejects the ride, the system looks for another taxi driver in the same area of the city.

The system offers programmatic interfaces (APIs) to enable the development of additional services on top of the basic one.

The system is provided with two optional modules:
\begin{description}
\item[Taxi reservation] allows the passenger to reserve a taxi by specifying the origin and the destination of the ride.
\item[Taxi sharing] allows the passengers to share a ride together, dividing the costs.
\end{description}
