The back-end stores its data in a RDBMS and  can run on every platform that supports the JVM.
The web application works on any web server that supports PHP. The mobile application is supported by Android, iOS. The system provides APIs to extend its functionalities, e.g. :
\begin{itemize}
\item taxi reservation
\item ride sharing
\item online payments
\item \ldots
\end{itemize}

%TODO block diagram

%  Alex
\subsection{User interfaces}
The user interfaces must provide the following logical characteristics both in the mobile app and in the web application:
\begin{itemize}
\item The possibility to choose the language used in every moment during every operation.
\item A first screen that lets the user login in order to begin operations.
\item A dashboard with links to every function in order to show the user the capabilities of the system and allow him to save time. 
\item A link to the dashboard in every screen.
\item A reminder in the top bar to show the last taxi service called, with a link to a screen which displays the reserved taxi history.
\end{itemize}

\subsection{Hardware interfaces} 
The system has to deal with the dichotomy of the web user interaction and the mobile one. It is necessary to provide a common look and feel, without losing simplicity with the different hardware interfaces. For instance, the compilation of data fields has to be made with multiple choices in order to simplify the user's experience of the app. Same goes for the dimension of buttons that can not be too small.

\subsection{Software interfaces}
The required software products used by the systems are:
\begin{itemize}
\item MySQL 5.7   \url{http://dev.mysql.com}
\item Java SE 8   \url{http://java.com}
\end{itemize}
% TODO review Software interfaces and user interfaces for avoiding inconsistency