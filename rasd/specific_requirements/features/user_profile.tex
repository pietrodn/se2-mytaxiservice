\label{user-profile}
\subsubsection{Purpose}
Any subscribed user can view or update his profile information. 

The system allows taxi drivers to use the service only if they load a new medical certificate every two years. A reminder e-mail is sent to the taxi driver three months before the expiration. %licence or medical certificate?

Logged passengers can:
\begin{itemize}
\item view their user profile
\item modify their user profile
\item view the latest taxi request
\item view the list of taxi requests
\end{itemize}

Logged taxi drivers can:
\begin{itemize}
\item view their user profile
\item modify their user profile
\item view the latest accepted taxi request
\item view the list of accepted requests
\end{itemize}

\subsubsection{Scenario 1}
Alice, a myTaxiService passenger without a car, wants to count how many times she has used a taxi in the last month.
She opens the home page of myTaxiService page on the web site and clicks on "login". 
After she has logged in correctly, she clicks on "load profile" and all the information about her account appears on screen, including the taxi request list. 

\subsubsection{Scenario 2}
Gabriele's girlfriend has discovered his password but he doesn't want her to know where he has been. So he decides to change his password immediately. He opens the app on her cell phone, he selects "load profile" and after he selects "modify password". The system asks him to enter the old password and the new one two times to avoid errors. Once he has verified that everything's ok, he selects "done", and the system confirms that the password has been successfully set.

\subsubsection{Scenario 3}
A recall mail is sent to Bob, a myTaxiService user that has registered as taxi-driver, to notice him that he has to update his profile with a new medical certificate. A week later, after the medical checkup that ensures his good health, he opens the app on his cell phone, he selects "load profile", then "modify". He enters the new medical certificate info, then confirms selecting "done". The system confirms that the medical certificate has been successfully inserted.

\subsubsection{Associated functional requirements}
\begin{enumerate}
\item The system must verify the consistence of the modified information. There mustn't be another one already subscribed with the same credentials. %TODO: it's possible to change username/email? That's not clear!
\item The system accepts a new password that contains al least one number and one capital letter and that has a minimum length of eight characters.
\item The system accept a new password only if the old one has been submitted correctly before.
\item The system asks for the new password two times, when a user wants to change it.
\item The system accepts the new password only if the same one was entered twice.
\item The system must allow the user to abort the modification at any time.
\item The modified item isn't saved successfully until the "done" choice isn't selected by the user. %TODO what's the "item"?
\item The system sends a recall mail three months before the expiration of the two years' validity of the last medical certificate submission.
\end{enumerate}




