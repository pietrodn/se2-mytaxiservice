\subsubsection{Purpose}

Any subscribed passenger shall be able to reserve a taxi for a ride at a predefined time. The passenger has to specify in advance the origin and the destination of the ride, along with the starting date and time.

\subsubsection{Scenario 1}
John McClane will need a taxi to get to the airport tomorrow morning. He opens the web application of myTaxiService and decides to book a taxi for 6:00 AM for a ride from his home to the airport. He confirms the request.

The morning after, at 5:50 AM, the first taxi driver in the queue gets McClane's request and accepts it. He comes to pick up McClane and brings him to the airport.

\subsubsection{Response sequence}
% Sequence diagrams and use cases

\subsubsection{Associated functional requirements}
\begin{enumerate}
\item The system presents the passenger with the option to reserve a taxi.
\item The system asks the passenger the origin and the destination of the ride.
\item Origin and destination must be valid addresses.
\item If GPS info is present and accurate within 50 m, the passenger can specify "current position" as the destination of the ride.
\item The system asks passenger for the date and time of the ride.
\item The system lets the passenger enter only valid dates and times.
\item The system lets the passenger reserve a taxi from 48 hours to 2 hours before the actual ride time.
\item 10 minutes before the specified arriving time, the system allocates a taxi for the passenger by putting a request in the queue as described in subsection~\ref{standard-call}.
\item After the request is accepted, the passenger gets notified with the ETA of the incoming taxi along with its position.
\end{enumerate}