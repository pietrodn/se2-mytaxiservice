\section{Elements to be Integrated}
\label{sec:elements}

In the Design Document~\cite{mytaxi-dd} we outlined four major high-level components, corresponding to the tiers of the system, which from now on will be referred as \textbf{subsystems}:
\begin{itemize}
    \item The database tier, which consists in the DBMS (not part of the software to be developed, but has to be integrated).
    \item Business tier, containing the server logic.
    \item Web tier, implementing the web interface.
    \item Client tier, which consists of desktop web browsers and our mobile application.
\end{itemize}

The integration process of our software is performed on two levels.
\begin{enumerate}
    \item integration of the different components (classes, Java Beans) inside the same subsystem;
    \item integration of different subsystems.
\end{enumerate}

The second step needs to be performed only for the component which contain the pieces of software that we are going to develop, namely the business tier, the mobile application in the client tier and part of the web tier (which, as stated in the DD, uses JSF to implement the web application).

% TODO do we need to list every single component?
