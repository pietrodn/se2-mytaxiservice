\section{Elements to be Integrated}
\label{sec:elements}

In the Design Document~\cite[p.~6]{mytaxi-dd} we outlined four major high-level components, corresponding to the tiers of the system, which – from now on – will be referred as \textbf{subsystems}:
\begin{description}
    \item[Database tier.] This is the DBMS; it's not part of the software to be developed, but has to be integrated.
    \item[Business tier.] This subsystem implements all the application logic and communicates with the front-ends.
    \item[Web tier.] The web tier implements the web interface and communicates with the business tier and the client browsers.
    \item[Client tier.] The client tier consists of desktop web browsers and of our mobile application.
\end{description}

The integration process of our software is performed on two levels.
\begin{enumerate}
    \item integration of the different components (classes, Java Beans) inside the same subsystem;
    \item integration of different subsystems.
\end{enumerate}

The first step needs to be performed only for the component which contain the pieces of software that we are going to develop, namely the business tier, the mobile application in the client tier and part of the web tier (which, as stated in the DD, uses JSF to implement the web application).
