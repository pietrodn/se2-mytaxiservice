\chapter{Individual Steps and Test Description}
\label{chap:individual-steps}

This chapter describes the individual test cases to be executed.
Each test case is identified with a code and is directly mapped with \autoref{tab:components-integration} for the integration between components and with \autoref{tab:subsystem-integration} for the integration between subsystems.

Test cases whose code starts with \textbf{SI} are integration tests between subsystems; test cases whose code starts with \textbf{I} are integration tests between components.

\section{Integration test case SI1}

\begin{tabular}{l p{0.7\textwidth}}
    \hline
    \textbf{Test Case Identifier} & SI1T1\\
    \hline
    \textbf{Test Item(s)} & Business tier $\rightarrow$ Database Tier\\
    \hline
    \textbf{Input Specification} & Typical calls to the methods of the JPA Entities, mapped with tables in the Database tier.\\
    \hline
    \textbf{Output Specification} & The Database tier shall respond by doing the correct queries on the test database. It must also react in the right way both if the requests are made correctly and if they come from unauthorized sources that are trying to access the data.\\
    \hline
    \textbf{Environmental Needs} & Complete implementation of the Java Entity Beans, Java Persistence API, Test Database, driver that calls the Java Entity Beans. \\
    \hline
    \textbf{Test Description} & The response will be compared with the expected output of the queries.\\
    \hline
    \textbf{Testing Method} & Automated with JUnit. \\
    \hline
\end{tabular}

\vspace{2em}

\section{Integration test case SI2}
\label{sec:performance-business}

\begin{tabular}{l p{0.7\textwidth}}
    \hline
    \textbf{Test Case Identifier} & SI2T1\\
    \hline
    \textbf{Test Item(s)} & Mobile application $\rightarrow$ Business Tier\\
    \hline
    \textbf{Input Specification} & Typical API calls (both correct and intentionally invalid ones) to the business tier (REST API).\\
    \hline
    \textbf{Output Specification} & The business tier shall respond accordingly to the API specification. Also, it must react correctly if the requests are malformed or maliciously crafted.\\
    \hline
    \textbf{Environmental Needs} & Complete implementation of the Business tier; REST API client (driver) that mocks the actual mobile client. \\
    \hline
    \textbf{Test Description} & The clients should make typical API calls to the business tier; the responses are then evaluated and checked against the expected output. The driver of this test is a standard REST API client that runs on Java.\\
    \hline
    \textbf{Testing Method} & Automated with JUnit. \\
    \hline
\end{tabular}

\vspace{2em}

\noindent\begin{tabular}{l p{0.7\textwidth}}
    \hline
    \textbf{Test Case Identifier} & SI2T2\\
    \hline
    \textbf{Test Item(s)} & Mobile application $\rightarrow$ Business Tier\\
    \hline
    \textbf{Input Specification} & Multiple concurrent (typical, correct) requests to the REST API of the business tier.\\
    \hline
    \textbf{Output Specification} & The business tier must answer the requests in a reasonable time with the applied load. \\
    \hline
    \textbf{Environmental Needs} & GlassFish Server, fully developed business tier, Apache JMeter.\\
    \hline
    \textbf{Test Description} & This test case assesses whether the business tier fulfills the performance requirement stated in the RASD~\cite{mytaxi-rasd} (section 3.3, \emph{Performance requirements}). In particular, the server has to support at least 1000 connected passengers at once, 95\% of requests shall be processed in less than 5 s and 100\% of requests shall be processed in less than 10 s.\\
    \hline
    \textbf{Testing Method} & Automated with Apache JMeter. \\
    \hline
\end{tabular}

\vspace{2em}

\section{Integration test case SI3}
\label{sec:performance-web}

\begin{tabular}{l p{0.7\textwidth}}
    \hline
    \textbf{Test Case Identifier} & SI3T1\\
    \hline
    \textbf{Test Item(s)} & Web tier $\rightarrow$ Business tier\\
    \hline
    \textbf{Input Specification} & Requests for services offered by the business tier, also invalid ones.\\
    \hline
    \textbf{Output Specification} & The web tier must call the proper REST APIs or report an error. \\
    \hline
    \textbf{Environmental Needs} & GlassFish Server, Web tier.\\
    \hline
    \textbf{Test Description} & This test has to ensure the right translation from HTTPS requests into REST APIs calls, reporting errors when needed.\\
    \hline
    \textbf{Testing Method} & Automated with JUnit. \\
    \hline
\end{tabular}

\vspace{2em}

\noindent\begin{tabular}{l p{0.7\textwidth}}
    \hline
    \textbf{Test Case Identifier} & SI3T2\\
    \hline
    \textbf{Test Item(s)} & Web tier $\rightarrow$ Business tier\\
    \hline
    \textbf{Input Specification} & Multiple concurrent API calls to the Business tier.\\
    \hline
    \textbf{Output Specification} & Web requests should be served without problems when a reasonable load is applied on the Business tier. \\
    \hline
    \textbf{Environmental Needs} & GlassFish Server, Web tier, Apache JMeter.\\
    \hline
    \textbf{Test Description} & This test case assesses whether the business tier fulfills the performance requirement stated in the RASD~\cite{mytaxi-rasd} (section 3.3, \emph{Performance requirements}). In particular, the system has to support at least 1000 connected passengers at once, 95\% of requests shall be processed in less than 5 s and 100\% of requests shall be processed in less than 10 s.\\
    \hline
    \textbf{Testing Method} & Automated with Apache JMeter. \\
    \hline
\end{tabular}

\vspace{2em}

\section{Integration test case SI4}

\begin{tabular}{l p{0.7\textwidth}}
    \hline
    \textbf{Test Case Identifier} & SI4T1\\
    \hline
    \textbf{Test Item(s)} & Client browser $\rightarrow$ Web tier\\
    \hline
    \textbf{Input Specification} & Typical and well-formed HTTPS requests from client browser; incomplete, malformed and maliciously crafted requests.\\
    \hline
    \textbf{Output Specification} & The web tier shall display the requested pages if the requests are valid; if the requests are invalid it shall display a generic error message. \\
    \hline
    \textbf{Environmental Needs} & GlassFish Server, fully developed web tier, HTTP client (driver).\\
    \hline
    \textbf{Test Description} & This test should emulate HTTP requests from typical users of the service and also incorrect requests.\\
    \hline
    \textbf{Testing Method} & Automated with JUnit. \\
    \hline
\end{tabular}

\vspace{2em}

\noindent\begin{tabular}{l p{0.7\textwidth}}
    \hline
    \textbf{Test Case Identifier} & SI4T1\\
    \hline
    \textbf{Test Item(s)} & Client browser $\rightarrow$ Web tier\\
    \hline
    \textbf{Input Specification} & Multiple concurrent requests to the web server.\\
    \hline
    \textbf{Output Specification} & Web pages should be served without problems when a reasonable load is applied on the web server. \\
    \hline
    \textbf{Environmental Needs} & GlassFish Server, fully developed web tier, Apache JMeter.\\
    \hline
    \textbf{Test Description} & This test case assesses whether the web tier fulfills the performance requirement stated in the RASD~\cite{mytaxi-rasd} (section 3.3, \emph{Performance requirements}). In particular, the web tier has to support at least 1000 connected passengers at once, 95\% of requests shall be processed in less than 5 s and 100\% of requests shall be processed in less than 10 s.\\
    \hline
    \textbf{Testing Method} & Automated with Apache JMeter. \\
    \hline
\end{tabular}

\newpage

\section{Integration test cases I01, I02, I03, I04, I05: components that integrate with the DBMS}

The following test cases refer to the integration between the Java Entity Beans and the underlying Database tier.
Since the test cases are very similar, they are grouped together.

\vspace{2em}

\begin{tabular}{l p{0.7\textwidth}}
    \hline
    \textbf{Test Case Identifier} & I01T1\\
    \hline
    \textbf{Test Item(s)} & TaxiLog $\rightarrow$ DBMS \\
    \hline
    \textbf{Input Specification} & Typical queries on table TaxiLog. \\
    \hline
    \hline
    \textbf{Test Case Identifier} & I02T1\\
    \hline
    \textbf{Test Item(s)} & Ride $\rightarrow$ DBMS \\
    \hline
    \textbf{Input Specification} & Typical queries on table Ride. \\
    \hline
    \hline
    \textbf{Test Case Identifier} & I03T1\\
    \hline
    \textbf{Test Item(s)} & User $\rightarrow$ DBMS \\
    \hline
    \textbf{Input Specification} & Typical queries on table User. \\
    \hline
    \hline
    \textbf{Test Case Identifier} & I04T1\\
    \hline
    \textbf{Test Item(s)} & Passenger $\rightarrow$ DBMS \\
    \hline
    \textbf{Input Specification} & Typical queries on table Passenger. \\
    \hline
    \hline
    \textbf{Test Case Identifier} & I05T1\\
    \hline
    \textbf{Test Item(s)} & TaxiDriver $\rightarrow$ DBMS \\
    \hline
    \textbf{Input Specification} & Typical queries on table TaxiDriver. \\
    \hline
    \hline
    \textbf{Output Specification} & The queries return the correct results. \\
    \hline
    \textbf{Environmental Needs} & GlassFish server, Test Database, driver for the Java Entity Beans. \\
    \hline
    \textbf{Test Description} & The purpose of these tests is to check that the correct methods of the Entity Beans are called, and that they execute the correct queries to the DBMS.  \\
    \hline
    \textbf{Testing Method} & Automated with JUnit. \\
    \hline
\end{tabular}

\vspace{2em}

\section{Integration test case I6}

\begin{tabular}{l p{0.7\textwidth}}
    \hline
    \textbf{Test Case Identifier} & I6T1\\
    \hline
    \textbf{Test Item(s)} & TaxiManager $\rightarrow$ TaxiQueueManager, TaxiDriver \\
    \hline
    \textbf{Input Specification} & Methods call from TaxiManager to TaxiQueueManager, to update driver's status and position and to find an available taxi in a specified TaxiZone.\\
    \hline
    \textbf{Output Specification} & The driver's position must be correctly updated without duplicating elements and the correct first available taxi must be returned and removed from the queue. The management of the driver's status must be properly handled.\\
    \hline
    \textbf{Environmental Needs} & GlassFish Server.\\
    \hline
    \textbf{Test Description} & The test aims to verify that the TaxiManager requests are correctly satisfied by TaxiQueueManager.\\
    \hline
    \textbf{Testing Method} & Automated with JUnit.\\
    \hline
\end{tabular}

\vspace{2em}

\section{Integration test case I7}

\begin{tabular}{l p{0.7\textwidth}}
    \hline
    \textbf{Test Case Identifier} & I7T1\\
    \hline
    \textbf{Test Item(s)} & RideManager $\rightarrow$ HistoryManager, User, Passenger, TaxiLog, Ride \\
    \hline
    \textbf{Input Specification} & Methods call from RideManager to HistoryManager, to manage and update the information of the rides.\\
    \hline
    \textbf{Output Specification} & The rides information must be correct and up-to-date.\\
    \hline
    \textbf{Environmental Needs} & GlassFish Server.\\
    \hline
    \textbf{Test Description} & Verify that the information is correctly updated and that it refers to the correct ride. Control that the rides' information is persistently updated.\\
    \hline
    \textbf{Testing Method} & Automated with JUnit.\\
    \hline
\end{tabular}

\vspace{2em}

\section{Integration test case I8}

\begin{tabular}{l p{0.7\textwidth}}
    \hline
    \textbf{Test Case Identifier} & I8T1\\
    \hline
    \textbf{Test Item(s)} & RideManager $\rightarrow$ ReservedRidePlugin\\
    \hline
    \textbf{Input Specification} & Calls to plugin methods to assure correct integration of the plugin.\\
    \hline
    \textbf{Output Specification} & The new functionalities of the plugin must be properly offered.\\
    \hline
    \textbf{Environmental Needs} & GlassFish Server.\\
    \hline
    \textbf{Test Description} & Assure that a ride can be reserved for a future time.\\
    \hline
    \textbf{Testing Method} & Automated with JUnit. \\
    \hline
\end{tabular}

\vspace{2em}

\section{Integration test case I9}

\begin{tabular}{l p{0.7\textwidth}}
    \hline
    \textbf{Test Case Identifier} & I9T1\\
    \hline
    \textbf{Test Item(s)} & RideManager $\rightarrow$ SharedRidePlugin\\
    \hline
    \textbf{Input Specification} & Calls to plugin methods to assure correct integration of the plugin.\\
    \hline
    \textbf{Output Specification} & The new functionalities of the plugin must be properly offered.\\
    \hline
    \textbf{Environmental Needs} & GlassFish Server.\\
    \hline
    \textbf{Test Description} & Assure that a ride can be shared between multiple users and that a split fee is correctly computed.\\
    \hline
    \textbf{Testing Method} & Automated with JUnit.\\
    \hline
\end{tabular}

\vspace{2em}

\section{Integration test case I10}

\begin{tabular}{l p{0.7\textwidth}}
    \hline
    \textbf{Test Case Identifier} & I10T1\\
    \hline
    \textbf{Test Item(s)} & UserManager $\rightarrow$ EmailSender, User \\
    \hline
    \textbf{Input Specification} & Methods call from UserManager to the EmailSender in order to guarantee a right email authentication process.\\
    \hline
    \textbf{Output Specification} & The email authentication process must be correctly handled.\\
    \hline
    \textbf{Environmental Needs} & GlassFish Server, mocked e-mail sender and receiver.\\
    \hline
    \textbf{Test Description} & Assure that a user can properly verify his/her email address in order to start using the system functionalities. In order to do that, a mock email address manager which simulates the user behaviour is needed.\\
    \hline
    \textbf{Testing Method} & Automated with JUnit and Mockito.\\
    \hline
\end{tabular}

\vspace{2em}

\section{Integration test case I11}

\begin{tabular}{l p{0.7\textwidth}}
    \hline
    \textbf{Test Case Identifier} & I11T1\\
    \hline
    \textbf{Test Item(s)} & TaxiManagerContainer $\rightarrow$ TaxiManager \\
    \hline
    \textbf{Input Specification} & Requests for the TaxiManager SessionBeans.\\
    \hline
    \textbf{Output Specification} & The SessionBeans must be correctly assigned and  the concurrency between the request must be properly managed.\\
    \hline
    \textbf{Environmental Needs} & GlassFish Server.\\
    \hline
    \textbf{Test Description} & Multiple requests for the TaxiManager SessionBeans have to be simultaneously carried out, in order to ensure that the users have no concurrency trouble.\\
    \hline
    \textbf{Testing Method} & Automated with JUnit and Arquillian.\\
    \hline
\end{tabular}

\vspace{2em}

\section{Integration test case I12}

\begin{tabular}{l p{0.7\textwidth}}
    \hline
    \textbf{Test Case Identifier} & I12T1\\
    \hline
    \textbf{Test Item(s)} & RideManagerContainer $\rightarrow$ RideManager \\
    \hline
    \textbf{Input Specification} & Requests for the RideManager SessionBeans.\\
    \hline
    \textbf{Output Specification} & The SessionBeans must be correctly assigned and  the concurrency between the request must be properly managed.\\
    \hline
    \textbf{Environmental Needs} & GlassFish Server.\\
    \hline
    \textbf{Test Description} & Multiple requests for the RideManager SessionBeans have to be simultaneously carried out, in order to ensure that the users have no concurrency trouble.\\
    \hline
    \textbf{Testing Method} & Automated with JUnit and Arquillian.\\
    \hline
\end{tabular}

\vspace{2em}

\section{Integration test case I13}

\begin{tabular}{l p{0.7\textwidth}}
    \hline
    \textbf{Test Case Identifier} & I13T1\\
    \hline
    \textbf{Test Item(s)} & UserManagerContainer $\rightarrow$ UserManager \\
    \hline
    \textbf{Input Specification} & Requests for the UserManager SessionBeans.\\
    \hline
    \textbf{Output Specification} & The SessionBeans must be correctly assigned and  the concurrency between the request must be properly managed.\\
    \hline
    \textbf{Environmental Needs} & GlassFish Server.\\
    \hline
    \textbf{Test Description} & Multiple requests for the UserManager SessionBeans have to be simultaneously carried out, in order to ensure that the users have no concurrency trouble.\\
    \hline
    \textbf{Testing Method} & Automated with JUnit and Arquillian.\\
    \hline
\end{tabular}

\vspace{2em}

\section{Integration test case I14}

\begin{tabular}{l p{0.7\textwidth}}
    \hline
    \textbf{Test Case Identifier} & I14T1\\
    \hline
    \textbf{Test Item(s)} & Controller $\rightarrow$ TaxiManagerContainer, RideManagerContainer, UserManagerContainer \\
    \hline
    \textbf{Input Specification} & Requests from Controller to the containers for the functionalities offered by SessionBeans within containers.\\
    \hline
    \textbf{Output Specification} & The controller has to be able to provide the right functionality carrying out the proper request to the containers.\\
    \hline
    \textbf{Environmental Needs} & GlassFish Server.\\
    \hline
    \textbf{Test Description} & Ensure that the controller is able to provide the functionalities of the system offered by the containers.\\
    \hline
    \textbf{Testing Method} & Automated with JUnit and Arquillian.\\
    \hline
\end{tabular}

\vspace{2em}

\section{Integration test case I15}

\begin{tabular}{l p{0.7\textwidth}}
    \hline
    \textbf{Test Case Identifier} & I15T1\\
    \hline
    \textbf{Test Item(s)} & UIManager $\rightarrow$ UIKit \\
    \hline
    \textbf{Input Specification} & Methods call from UIManager to the UI elements, to display output data and change their status.\\
    \hline
    \textbf{Output Specification} & The view shall change accordingly and display the output data.\\
    \hline
    \textbf{Environmental Needs} & Xcode, iOS Simulator.\\
    \hline
    \textbf{Test Description} & Verify that the bindings of the view items are correctly set in the controller and that the view actually changes and responds to method calls. Check that the output is displayed correctly.\\
    \hline
    \textbf{Testing Method} & Automated (iOS testing suite), manual testing on physical devices. \\
    \hline
\end{tabular}

\vspace{2em}

\noindent\begin{tabular}{l p{0.7\textwidth}}
    \hline
    \textbf{Test Case Identifier} & I15T2\\
    \hline
    \textbf{Test Item(s)} & UIManager $\rightarrow$ UIKit \\
    \hline
    \textbf{Input Specification} & Perform (or simulate) gestures on the UI elements.\\
    \hline
    \textbf{Output Specification} & The controller shall receive the actions and log them. \\
    \hline
    \textbf{Environmental Needs} & Xcode, iOS Simulator.\\
    \hline
    \textbf{Test Description} & Check that the gestures perform the correct actions on the controller.\\
    \hline
    \textbf{Testing Method} & Automated (iOS testing suite), manual testing on physical devices. \\
    \hline
\end{tabular}

\vspace{2em}

\section{Integration test case I16}

\begin{tabular}{l p{0.7\textwidth}}
    \hline
    \textbf{Test Case Identifier} & I16T1\\
    \hline
    \textbf{Test Item(s)} & UIManager $\rightarrow$ android.view \\
    \hline
    \textbf{Input Specification} & Methods call from UIManager to the UI elements, to display output data and change their status.\\
    \hline
    \textbf{Output Specification} & The view shall change accordingly and display the output data.\\
    \hline
    \textbf{Environmental Needs} & Android Emulator. \\
    \hline
    \textbf{Test Description} & Verify that the bindings of the view items are correctly set in the controller and that the view actually changes and responds to method calls. Check that the output is displayed correctly.\\
    \hline
    \textbf{Testing Method} & Automated (Android testing suite), manual testing on physical devices. \\
    \hline
\end{tabular}

\vspace{2em}

\noindent\begin{tabular}{l p{0.7\textwidth}}
    \hline
    \textbf{Test Case Identifier} & I16T2\\
    \hline
    \textbf{Test Item(s)} & UIManager $\rightarrow$ android.view \\
    \hline
    \textbf{Input Specification} & Perform (or simulate) gestures on the UI elements.\\
    \hline
    \textbf{Output Specification} & The controller shall receive the actions and log them. \\
    \hline
    \textbf{Environmental Needs} & Android Emulator. \\
    \hline
    \textbf{Test Description} & Check that the gestures perform the correct actions on the controller.\\
    \hline
    \textbf{Testing Method} & Automated (Android testing suite), manual testing on physical devices. \\
    \hline
\end{tabular}

\vspace{2em}

\section{Integration test case I17}

\begin{tabular}{l p{0.7\textwidth}}
    \hline
    \textbf{Test Case Identifier} & I17T1\\
    \hline
    \textbf{Test Item(s)} & GPSManager $\rightarrow$ CoreLocation \\
    \hline
    \textbf{Input Specification} & Calls to the CoreLocation framework methods to get location data of the user. \\
    \hline
    \textbf{Output Specification} & User location data or a meaningful error status shall be returned. \\
    \hline
    \textbf{Environmental Needs} & Xcode, iOS Simulator. \\
    \hline
    \textbf{Test Description} & The purpose of the test is to check that our controller (GPSManager) can correctly get the position from the corresponding iOS API. Error statuses shall also be checked. \\
    \hline
    \textbf{Testing Method} & Automated (iOS testing suite). \\
    \hline
\end{tabular}

\vspace{2em}

\section{Integration test case I18}

\begin{tabular}{l p{0.7\textwidth}}
    \hline
    \textbf{Test Case Identifier} & I18T1\\
    \hline
    \textbf{Test Item(s)} & GPSManager $\rightarrow$ LocationListener \\
    \hline
    \textbf{Input Specification} & Calls to the Android Location framework methods to get location data of the user. \\
    \hline
    \textbf{Output Specification} & User location data shall be returned, or a meaningful error status. \\
    \hline
    \textbf{Environmental Needs} & Android Emulator. \\
    \hline
    \textbf{Test Description} & The purpose of the test is to check that our controller (GPSManager) can correctly get the position from the corresponding Android API. Error statuses shall also be checked. \\
    \hline
    \textbf{Testing Method} & Automated (Android testing suite). \\
    \hline
\end{tabular}

\vspace{2em}

\section{Integration test case I19}

\begin{tabular}{l p{0.7\textwidth}}
    \hline
    \textbf{Test Case Identifier} & I19T1\\
    \hline
    \textbf{Test Item(s)} & UIManager $\rightarrow$ GPSManager \\
    \hline
    \textbf{Input Specification} & Calls to GPSManager methods to get the user's location. \\
    \hline
    \textbf{Output Specification} & The location data shall be returned from GPSManager in a suitable format, or an exception shall be raised if the location data is not available. \\
    \hline
    \textbf{Environmental Needs} & Xcode, iOS Simulator, Android Emulator.\\
    \hline
    \textbf{Test Description} & GPSManager should be able to return the correct GPS data in a universal and consistent format independently from the architecture (iOS or Android).\\
    \hline
    \textbf{Testing Method} & Automated (Android and iOS testing suites). \\
    \hline
\end{tabular}

\vspace{2em}

\noindent\begin{tabular}{l p{0.7\textwidth}}
    \hline
    \textbf{Test Case Identifier} & I19T2\\
    \hline
    \textbf{Test Item(s)} & UIManager $\rightarrow$ ResourceLoader \\
    \hline
    \textbf{Input Specification} & Load application resources (images, sounds, data) from ResourceManager. \\
    \hline
    \textbf{Output Specification} & ResourceManager should provide the required resources without errors. \\
    \hline
    \textbf{Environmental Needs} & Xcode, iOS Simulator, Android Emulator.\\
    \hline
    \textbf{Test Description} & ResourceLoader is responsible for the retrieval of the resources stored into the application bundle.
    This test aims to assessing that all the resources can be accessed without errors by the mobile application. \\
    \hline
    \textbf{Testing Method} & Automated (Android and iOS testing suites).\\
    \hline
\end{tabular}

\vspace{2em}

\section{Integration test case I20}

\begin{tabular}{l p{0.7\textwidth}}
    \hline
    \textbf{Test Case Identifier} & I20T1\\
    \hline
    \textbf{Test Item(s)} & WebController $\rightarrow$ JavaServerFaces \\
    \hline
    \textbf{Input Specification} & WebController is given the typical output to be displayed on the web page.\\
    \hline
    \textbf{Output Specification} & JavaServerFaces shall display the required output in a correct way.\\
    \hline
    \textbf{Environmental Needs} & GlassFish Server, Stub of the Business Tier to provide the output data. \\
    \hline
    \textbf{Test Description} & The purpose of this test case is to check if JSF can communicate correctly with the WebController bean.\\
    \hline
    \textbf{Testing Method} & Automated with JUnit. \\
    \hline
\end{tabular}

\vspace{2em}

\section{Integration test case I21}

\begin{tabular}{l p{0.7\textwidth}}
    \hline
    \textbf{Test Case Identifier} & I21T1\\
    \hline
    \textbf{Test Item(s)} & WebContainer $\rightarrow$ WebController \\
    \hline
    \textbf{Input Specification} & Run the web application.\\
    \hline
    \textbf{Output Specification} & WebContainer injects the WebController bean, using JSF.\\
    \hline
    \textbf{Environmental Needs} & GlassFish Server.\\
    \hline
    \textbf{Test Description} & This test verifies if the correct component is injected into JSF.\\
    \hline
    \textbf{Testing Method} & Automated with JUnit. \\
    \hline
\end{tabular}
